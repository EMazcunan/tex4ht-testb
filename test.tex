%% make4ht -us -c config -f html5+tidy+dvisvgm_hashes -e build.mk4 test.tex 


\documentclass[12pt,a4paper,twoside]{article}
%\ifdefined\HCode\def\pgfsysdriver{pgfsys-dvisvgm4ht.def}\fi % sale la caja con el título, pero con xx en lugar de acentos y da WARNING sobre Unbalanced Tag (/p)
%\ifdefined\HCode\def\pgfsysdriver{/home/eva/texmf/tex/latex/dvisvgm/pgfsys-dvisvgm4ht.def}\fi 
%\ifdefined\HCode\def\pgfsysdriver{pgfsys-tex4ht-updated.def}\fi % no se visualiza .svg (texto alternativo picture)


\usepackage[utf8]{inputenx}
%\usepackage[utf8]{inputenc}
\usepackage[T1]{fontenc}
\usepackage{lmodern}
\usepackage[
	spanish, 
	es-minimal, 
	%es-sloppy
	%es-nolists, % desactiva los cambios en los símbolos de enumerate e itemize
	%es-noindentfirst, % el primer párrafo después de cada sección sí se sangra
	%es-nosectiondot, % desactiva punto después de número sección (en texto y toc)
	%es-noquoting, % desactiva abreviaciones << >> y "` "' del entorno quoting
	%es-notilde,% desactiva abreviaciones ~- ~-- ~---
	%es-ucroman, % desactiva redefinición de \roman para dar i, ii, ... en versalitas
	%es-noshorthands,
	%es-lcroman,
]{babel}
%\renewcommand\shorthandsspanish{`}
%\addto\shorthandspanish{\spanishdeactivate{"~<>}}
%\deactivatequoting
\unaccentedoperators

\usepackage{lipsum}

\usepackage{amsmath}
\usepackage{amsthm}

%\usepackage{tikz}
%%\usetikzlibrary{babel}
%
%\usepackage{tcolorbox}

%% ========================
%\usepackage{mytcolorbox}
%\usepackage{exsol}
%\usepackage{scq}


%% ========================

\usepackage[%
%tex4ht
]{hyperref}

\title{De LaTeX a HTML}
\author{Nombre español}

\begin{document}

\maketitle



\ifdefined\HCode\else
	\tableofcontents
\fi


%
%\section{exsol}
%
%\begin{ex}[sol later]
%Enunciado ejercicio
%\begin{sol}
%Solución
%\end{sol}
%\end{ex}
%
%
%\section{scq}
%En PDF compilar dos/tres veces hasta que desaparezcan las marcas 
%
%\begin{scq}
%	¿Cuánto vale $3^2$?%
%	\begin{choice}[x]
%		$9$	
%	\end{choice}	
%	\begin{choice}
%		$6$	
%	\end{choice}
%	\begin{choice}
%		$0$ 
%	\end{choice}
%	\begin{feedback}
%		$3^2=3\times 3 = 9$.
%	\end{feedback}
%\end{scq}

%\end{document}
%%%%
%\section{tcolorbox}
%
%\begin{tcolorbox}[%
%	title = {Título}
%]
%Contenidos 
%\end{tcolorbox}
%
%\lipsum[1]
%%%\end{document}

\section{Referencias cruzadas}


Referencia al número de la última sección: \ref{sec:last} %y \ref{sec:prelast}

\section{Fuentes}

Escritura en \textbf{negrita}, \textit{itálica}, \textsf{sans serif} y \texttt{teletype}.

%Declaraciones: Escritura en {\bfseries negrita}, {\itshape itálica}, {\sffamily sans serif} y {\ttfamily teletype}.
%\end{document}
\section{Matemáticas}

\input{samples/maths-sample.tex}

%\end{document}
\section{Sección}

\lipsum[1-15]

\subsection{Subsección}

\lipsum[1-15]

\subsection{Otra subsección}

\lipsum[1-15]		

\section{Otra sección}

\lipsum[1-15]	

\section*{Otra sección no numerada}
\ifdefined\HCode\else
\addcontentsline{toc}{section}{Otra sección no numerada}
\fi

\lipsum[1-15]			

\section{Una última sección con el título suficientemente largo como para provocar un salto de línea}
\label{sec:last}
\lipsum[1-15]	

%\section{Soluciones}
%
%\printSolutionsLater{}{sol}

\end{document}